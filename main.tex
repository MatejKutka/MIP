% Metódy inžinierskej práce

\documentclass[10pt,oneside,slovak,a4paper]{article}

\usepackage[slovak]{babel}
%\usepackage[T1]{fontenc}
\usepackage[IL2]{fontenc} 
\usepackage[utf8]{inputenc}
\usepackage{graphicx}
\usepackage{url} 
\usepackage{hyperref} 

\usepackage{cite}


\pagestyle{headings}

\title{Názov\thanks{Semestrálny projekt v predmete Metódy inžinierskej práce, ak. rok 2022/23, vedenie: Ing. Fedor Lehocki, Phd.}} 

\author{Matej Kuťka\\[2pt]
	{\small Slovenská technická univerzita v Bratislave}\\
	{\small Fakulta informatiky a informačných technológií}\\
	{\small \texttt{xkutka@stuba.sk}}
	}

\date{\small 6. november 2022}

\begin{document}

\maketitle

\begin{abstract}
\ldots
\end{abstract}

\section{Úvod}

Gamifikácia je spôsob, ktorým sa prenášajú herné prvky zo sveta hier do reality. To znamená, že sa realita stáva viac hravou nielen pri relaxácii ale aj pri vážnejších záležitostiach. V tomto článku sa budem zaoberať využitím gamifikácie a teda herných prvkov pri učení cudzieho jazyka a to hlavne v aplikácii Duolingo, ktorá je veľmi dobrým príkladom praktického využitia gamifikácie v tejto téme. Chcel by som týmto článkom poukázať na možnú zmenu systému výučby nielen pri používaní aplikácií akou je Duolingo ale aj celkovo v školstve.

\section{Herné prvky Duolinga}
V Duolingu sú pri výučbe použité mnohé gamifikačné prvky, ktoré povzbudzujú užívateľov učiť sa pravidelne jazyk, ktorý si vybrali a motivujú ich splniť vybratý cieľ.

\subsection{Používanie aplikácie každý deň}
Na podporenie užívateľov na konsistenciu v ich výučbe je požitý Streak systém. Tento ukazuje koľko dní za sebou užívateľ splnil lekciu z vybraného jazyka. Takto sa podporuje činnosť užívateľa na aplikácii a dáva jeden z dôvodov, prečo používať aplikáciu každý deň.

\subsection{Odmeny}
Odmenou v Duolingu je mena pod názvom Lingot. Túto odmenu je možné použiť v obchode umiestnenom priamo v aplikácii. Je možné ich vymeniť za Streak Freeze, ktorý umožňuje zachovať si streak z predchádzajúcich dní aj pri vynechaní jedného dňa alebo zaplatiť 5 lingotov na motivovanie samého seba. Motivuje sa užívateľ tým spôsobom, že ak si neudrží 7 denný streak tak lingoty stratí a keď to zvládne, lingoty si zdvojnásobí. 

\subsection{Systém úrovní}
Ďalším gamifikačným systémom je získavanie experience points alebo teda bodov skúsenosti po splnení cvičenia. Týmto spôsobom dostáva užívateľ aj niečo iné ako iba samotné vedomosti a ukazuje to taktiež dosiahnutý pokrok vo výučbe.

\subsection{Odznaky}
Sýstém získavanie odznakov podoruje v užívateľovi hlavne súťaživosť. Pomocou získaných bodov skúsenosti a teda s pomocou systému úrovní sú užívatelia rozmiestnený v tabuľke každý týždeň. Nový užívatelia začínajú na bronzovom odznaku (každý odznak má samostatnú tabuľku) a posúvajú sa podľa umiestnenia v tabuľke o odznak hore alebo aj dole. Miesto na ktorom musí byť užívateľ umiestnený  aby sa posunul vyššie v tabuľe je rôzne. V najnižšej bronzovej úrovni musí byť užívateľ v top 20-tich užívateľoch. Potom v striebornej top 15-tich, v zlatej top 10-tich. Potom v safírovej, rubínovej, smaragdovej, ametystovej a perlovej je potrebné byť v top 7 a nakoniec v obsidiánovej top 5. Najvyššou úrovnňou je diamantová, z ktorej  sa už postúpiť nedá.

\subsection{}
\cite{Huynh2016}V duolingu je taktiež možné sa porovnávať so svojimi kamarátmi na rebríčku a slúži t oako skvelá motivácia pre pokračovanie a vydržanie pri štúdiu.


\section{Dáta}
Duolingo v tomto čase poskytuje 40 možných jazykov a z toho 5 je v beta móde a jeden je pripravený na 6 percent. Z toho má najviac užívateľov španielčina s 31.4 miliónmi a najmenej Zulu so 132 tisícmi. Ako príklad som vybral japončinu, ktorá má v tomto čase 13 miliónov užívateľov. Tento kurz má 90 skupín, ktoré sa ďalej delia na menšie lekcie. Tieto je možné opakovať na precvičenie a taktiež ich týmto opakovaním vylepšovať. Štruktúra kurzu japončiny na začiatku je najprv naučenie sa písmen a ich výslovnosti. V každej z menších lekcií je skupina znakov, ktoré sa potom skladajú do slov. Postupne sa užívateľ prepacováva ku konverzáčným schopnostiam a tak zvyšuje svoje chápanie jazyka.

\section{Využitie v školách}
V školách je časté vyučovanie jazyka pomocou zastaralých metód. Učenie sa slovíčok naspamäť a podobné zastaralé metódy. Gamifikácia sa využíva iba veľmi zriedkavo. Využitie gamifikačných prvkov a aplikácií akou je duolingo alebo iných hravých aplikácií, ktorými môžu byť napríklad Kahoot. Takýmto spôsobom sa výučba obohatí a pridá niečo navyše k motivácii študentov. Samozrejme je potrebné vyučovať aj gramatiku a nespoliehať sa úplne na aplikácie poskytované na internete. Ale aj v tomto je možné uplatniť prvky gamifikácie. To však záleží iba na kreativite vyučujúceho.


\bibliography{bibliografia}
\bibliographystyle{plain}

\end{document}